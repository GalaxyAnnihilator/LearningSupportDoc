\documentclass{article}
\usepackage{graphicx} % Required for inserting images
\graphicspath{ {./images/} }
\graphicspath{ {images/} }
\usepackage{xcolor}
\usepackage{amsmath}
\usepackage{multicol}
\usepackage{amsmath}
\usepackage{amsfonts}
\usepackage{amssymb}
\usepackage{url}
\usepackage{mdframed}
\usepackage{hyperref}
\usepackage{subfigure}
\usepackage{fancybox,graphicx}
\usepackage{mathrsfs} 
\usepackage{amsfonts}
\usepackage{longtable,array}
\usepackage{multirow}
\usepackage[latin1]{inputenc}
\usepackage[T1]{fontenc}
\usepackage{array}
\usepackage{booktabs}

\usepackage{hyperref}
\newmdenv[linecolor=black,skipabove=\topsep,skipbelow=\topsep,
leftmargin=-5pt,rightmargin=-5pt,
innerleftmargin=5pt,innerrightmargin=5pt]{mybox}
\usepackage[a4paper,left=22mm,top=20mm,right=22mm]{geometry}

\title{Calculus II - Chapter 4\\Second-Order Ordinary Differential Equations}
\author{By Han Le from USTH LEARNING SUPPORT}
\date{June 2024}

\begin{document}

\maketitle
\tableofcontents
\newpage
\section{Simple Forms of Second-Order Differential Equations}
Any second-order differential equation that can be written in the form:
\begin{equation}
    \displaystyle\frac{d^2y}{dx^2}=F\left(x,y,\frac{dy}{dx}\right)
\end{equation}
\textbf{Note: }$y'=\displaystyle\frac{dy}{dx}$, $y''=\displaystyle\frac{d^2y}{dx^2}$
\subsection{Equations with the dependent variable $y$ missing}
\begin{mybox}
    If $y$ does not occur explicitly in the function F in equation (1), the Second-Order Differential Equation has the form:
    \begin{equation}
        \displaystyle\frac{d^2y}{dx^2}=F\left(x,\frac{dy}{dx}\right)
    \end{equation}
Let $z=\displaystyle\frac{dy}{dx}$, then $\displaystyle\frac{dz}{dx}=\frac{d}{dx}\left(\displaystyle\frac{dy}{dx}\right)=\frac{d^2y}{dx^2}$, equation (2) becomes:
    \begin{equation}
        \displaystyle\frac{dz}{dx}=F(x,z)
    \end{equation}
We can see that the Second-Order Equation (2) has become the First-Order Differential Equation (3).\\ 
Now, we solve the equation (3) to obtain $z$, then integrate $z=\displaystyle\frac{dy}{dx}$ to find $y$ - the solution for equation (2).
\end{mybox}
\textbf{Example:} Find the general solution of the following equation:
\begin{equation*}
    3\displaystyle\frac{d^2y}{dx^2}=\frac{dy}{dx}
\end{equation*}
\begin{center}
    \textbf{Solution}
\end{center}
We can see that the variable $y$ does not appear explicitly in the equation, there are only $y''$ and $y'$.\\
Let $z=\displaystyle\frac{dy}{dx} \Rightarrow \frac{dz}{dx}=\frac{d^2y}{dx^2}$\\
The equation becomes:
\begin{flalign*}
    &&3\displaystyle\frac{dz}{dx}&=z&&\\
    \iff&& 3dz&=zdx&& (*)
\end{flalign*}
- If $z=0 \Rightarrow y=C$ (C is an arbitrary constant) is a solution of the equation.\\
- If $z\ne 0$, we have:
\begin{flalign*}
    (*)\iff&& \displaystyle\frac{1}{z}dz&=\frac{1}{3}dx&&\\
    \iff&& \int \frac{1}{z}dz&=\int\frac{1}{3}dx&&\\
    \iff&& \ln(z)&=\frac{x}{3}+C\text{ (C is an arbitrary constant)}&&\\
    \iff&& e^{\ln(z)}&=e^{(x/3)+C}&&\\
    \iff&& z&=e^{x/3}.e^C&&\\
    \iff&& \frac{dy}{dx}&=C'e^{x/3} \left(C'=e^C\right)&&\\
    \iff&& \int dy&=\int C'e^{x/3}dx&&\\
    \iff&& y&=3C'\int e^{x/3}d\left(\frac{x}{3}\right)&&\\
    \iff&& y&=C_1e^{x/3}+C_2 \text{ (}C_1=3C'\text{ and } C_2 \text{ are arbitrary constants}\text{)}&&
\end{flalign*}
Note that solution $y=C$, where $C$ is an arbitrary constant, is included in the above solution.\\
Finally, $y=C_1e^{x/3}+C_2$, where $C_1,C_2$ are arbitrary constants, is the general solution of the equation.\\
\textbf{Example: }Find the general solution of the following equation:
\begin{equation*}
    (x-1)(x-2)y''= y'-1
\end{equation*}
\begin{center}
    \textbf{Solution}
\end{center}
Let $z=\displaystyle\frac{dy}{dx}$, the equation becomes:
\begin{flalign*}
    &&(x-1)(x-2)\displaystyle\frac{dz}{dx}&=z-1&&\\
    \iff&& \frac{dz}{dx}&=\frac{1}{(x-1)(x-2)}(z-1)&&\\
    \iff&& \frac{dz}{dx}&=\frac{1}{(x-1)(x-2)}z-\frac{1}{(x-1)(x-2)}&&\\
    \iff&& \frac{dz}{dx}-\frac{1}{(x-1)(x-2)}z&=-\frac{1}{(x-1)(x-2)}&& (*)
\end{flalign*}
\begin{flalign*}
    \text{Consider }&\displaystyle\displaystyle\int\frac{-1}{(x-1)(x-2)}dx&&\\
    =&\int\frac{x-2-x+1}{(x-1)(x-2)}dx&&\\
    =&\int\left(\frac{1}{x-1}-\frac{1}{x-2}\right)dx&&\\
    =&\ln|x-1|-\ln|x-2|&&\\
    =&\ln\left|\frac{x-1}{x-2}\right|&&
\end{flalign*}
$\Rightarrow e^{\ln[(x-1)/(x-2)]}=\displaystyle\frac{x-1}{x-2}$ is the integrating factor for (*)
\begin{flalign*}    
    \text{(*) }\iff&&\displaystyle\frac{x-1}{x-2}.\frac{dz}{dx}-\frac{x-1}{x-2}.\frac{z}{(x-1)(x-2)}&=-\frac{x-1}{x-2}.\frac{1}{(x-1)(x-2)}&&\\
\iff&&\frac{x-1}{x-2}.\frac{dz}{dx}-\frac{z}{(x-2)^2}&=-\frac{1}{(x-2)^2}\\
\iff&& d\left(\frac{x-1}{x-2}.z\right)&=-\frac{1}{(x-2)^2}&&\\
\iff&& \frac{x-1}{x-2}.z&=-\int\frac{1}{(x-2)^2}dx&&\\
\iff&& \frac{x-1}{x-2}.z&=-\int (x-2)^{-2}d(x-2)&&\\
\iff&& \frac{x-1}{x-2}.z&=(x-2)^{-1}+C_1 \text{ ($C_1$ is an arbitrary constant)}&&\\
\iff&& z&=\frac{1}{x-2}.\frac{x-2}{x-1}+C_1.\frac{x-2}{x-1}&&\\
\iff&& \frac{dy}{dx}&=\frac{1}{x-1}+C_1.\frac{x-1-1}{x-1}&&\\
\iff&& dy&=\left[\frac{1}{x-1}+C_1\left(1-\frac{1}{x-1}\right)\right]dx&&\\
\iff&& y&=\int\frac{1}{x-1}dx+\int C_1.\left(1-\frac{1}{x-1}\right)dx&&\\
\iff&& y&=\ln|x-1|+C_1(x-\ln|x-1|)+C_2 \text{ ($C_2$ is an arbitrary constant)}&&\\
\iff&& y&=(1-C_1)\ \ln|x-1| + C_1x+C_2
\end{flalign*}
Finally, $y=(1-C_1)\ \ln|x-1| + C_1x+C_2$, where $C_1,C_2$ are arbitrary constants, is the general solution of the equation.
\newpage
\subsection{Equations with the independent variable $x$ missing}
\begin{mybox}
    If $x$ does not occur explicitly in the function F in equation (1), the Second-Order Differential Equation has the form:
    \begin{equation}
        \displaystyle\frac{d^2y}{dx^2}=F\left(y,\frac{dy}{dx}\right)
    \end{equation}
Let $z=\displaystyle\frac{dy}{dx}$, then $\displaystyle\frac{d^2y}{dx^2}=\displaystyle\frac{dz}{dx}=\frac{dz}{dy}.\displaystyle\frac{dy}{dx}=z.\frac{dz}{dy}$, equation (4) becomes:
    \begin{equation}
        z.\displaystyle\frac{dz}{dy}=F(y,z)
    \end{equation}
We can see that the Second-Order Equation (4) has become the First-Order Differential Equation (5).\\ 
Now, we solve the equation (5) to obtain $z$, then integrate $z=\displaystyle\frac{dy}{dx}$ to find $y$ - the solution for equation (4).
\end{mybox}
\textbf{Example:} Find the general solution for the equation:
\begin{equation*}
    y''+2y^{-1}(y')^2=y'
\end{equation*}
\begin{center}
    \textbf{Solution}
\end{center}
We can see that the variable $x$ does not appear explicitly in the equation, there are only $y'',y'$ and $y$.\\
Let $z=\displaystyle\frac{dy}{dx}\Rightarrow z.\frac{dz}{dy}=\frac{d^2y}{dx^2}$\\
The equation becomes:
\begin{flalign*}
    &&\displaystyle z.\frac{dz}{dy}+2y^{-1}z^2=z&&(*)
\end{flalign*}
\text{- If $z=0$, then $y=C$,} \text{where $C$ is an arbitrary constant,}\text{ is a solution of the equation.}\\
- If $z\ne 0$, we have:
\begin{flalign*}
    (*)\iff&&\frac{dz}{dy} + \frac{2z}{y}=1&&(1)
\end{flalign*}
The integrating factor is:
\begin{flalign*}
    &&e^{\int (2/y)dy}=e^{2\ln(y)}=e^{\ln(y^2)}=y^2&&\\
    (1)\iff&&y^2\displaystyle\frac{dz}{dy}+\frac{2z}{y}.y^2&=y^2&&\\
    \iff&&y^2\displaystyle\frac{dz}{dy}+2yz&=y^2&&\\
    \iff&&d\left(y^2z\right)&=y^2&&\\
    \iff&&y^2z&=\int y^2dy&&\\
    \iff&&y^2z&=\frac{1}{3}y^3+C\text{ ($C$ is an arbitrary constant)}&&\\
    \iff&&z&=\frac{1}{3}y+\frac{C}{y^2}&&\\
    \iff&&\frac{dy}{dx}&=\frac{1}{3}y+\frac{C}{y^2}&&\\
    &&\text{(We can solve this equation}& \text{ by using Bernoulli equation)}\\
    \iff&&\frac{dy}{dx}-\frac{1}{3}y&=\frac{C}{y^2}&&\\
    \iff&&y^2\frac{dy}{dx}-\frac{1}{3}y^3&=C&& (2)
\end{flalign*}
Let $u=y^3\\
\Rightarrow du=3y^2dy\\
\Rightarrow \displaystyle\frac{du}{dx}=3y^2\frac{dy}{dx}\\
\Rightarrow \frac{dy}{dx}=\frac{1}{3y^2}.\frac{du}{dx}$
\begin{flalign*}
    (2)\iff&&y^2.\displaystyle\frac{1}{3y^2}.\frac{du}{dx}-\frac{1}{3}u&=C&&\\
    \iff&&\frac{1}{3}.\frac{du}{dx}-\frac{1}{3}u&=C&&\\
    \iff&&\frac{du}{dx}-u&=3C&&(3)
\end{flalign*}
The integrating factor is:
\begin{flalign*}
    &&e^{\int (-1)dx}=e^{-x}&&\\
    (3)\iff&&e^{-x}\frac{du}{dx}-e^{-x}u&=3Ce^{-x}\\
    \iff&&d\left(e^{-x}u\right)&=3Ce^{-x}\\
    \iff&& e^{-x}u&=-3Ce^{-x}+C_1 \text{ ($C_1$ is an arbitrary constant)}\\
    \iff&& u&=-3C+C_1e^x\\
    \iff&& y^3&=C_1e^x+C_2 \text{ ($C_2=-3C$)}\\
    \iff&& y&=\displaystyle\sqrt[3]{C_1e^x+C_2}
\end{flalign*}
Note that the first solution $y=C$, where $C$ is an arbitrary constant, is included in this solution.\\
Finally, $y=\displaystyle\sqrt[3]{C_1e^x+C_2}$, where $C_1,C_2$ are arbitrary constants, is the solution of the equation.
\section{Homogeneous Second-Order Linear Differential Equations}
\begin{mybox}
The standard form of a \textbf{Homogeneous Second-Order Linear Differential Equation} with constant-coefficients:
\begin{equation}
    y''+ ay' + by = 0
\end{equation}
where $a,b$ are constants.\\
The \textbf{characteristic equation} of the equation (6) is:
\begin{equation*}
    P(t)=t^2+at+b=0
\end{equation*}
\textbf{- Case 1:} $P(t)$ has two distinct real roots $t_1,t_2$ ($\Delta>0$), the general solution of equation (6) is:
\begin{equation*}
    y(x)=C_1y_1(x)+C_2y_2(x)=C_1e^{t_1x}+C_2e^{t_2x}
\end{equation*}
where $C_1,C_2$ are arbitrary constants and $y_1(x),y_2(x)$ are two linearly independent solutions.\\
\textbf{- Case 2:} $P(t)$ has a real double root $t_0$ ($\Delta=0$), the general solution of equation (6) is:
\begin{equation*}
    y(x)=C_1y_1(x)+C_2y_2(x)=C_1e^{t_0x}+C_2xe^{t_0x}
\end{equation*}
where $C_1,C_2$ are arbitrary constants and $y_1(x),y_2(x)$ are two linearly independent solutions.\\
\textbf{- Case 3:} $P(t)$ has two complex conjugate roots $t_1,t_2=\alpha\pm\omega i$ ($\Delta<0$), the general solution of equation (6) is:
\begin{equation*}
    y(x)=C_1y_1(x)+C_2y_2(x)=C_1e^{\alpha x}\cos(\omega x)+C_2e^{\alpha x}\sin(\omega x)
\end{equation*}
where $C_1,C_2$ are arbitrary constants and $y_1(x),y_2(x)$ are two linearly independent solutions.
\end{mybox}
\ \\\\\\\\
\textbf{Example: }Find the general solution of the differential equation
\begin{center}
    $y''-y'-6y=0$\\
    \textbf{Solution}
\end{center}
The characteristic equation is:
\begin{flalign*}
    \ \ &t^2-t-6=0&&\\
    \iff&\left[\begin{matrix}
        t=&3\\
        t=&-2
    \end{matrix}
    \right.\\
    \Rightarrow y&(x)=C_1e^{3x}+C_2e^{-2x} \text{, where $C_1,C_2$ are arbitrary constants, is the general solution of the equation.}
\end{flalign*}
\textbf{Example: }Find the general solution of the differential equation
\begin{center}
    $y''+3y'-4y=0$\\
    \textbf{Solution}
\end{center}
The characteristic equation is:
\begin{flalign*}
    \ \ &t^2+3t-4=0&&\\
    \iff&\left[\begin{matrix}
        t=&1\\
        t=&-4
    \end{matrix}
    \right.\\
    \Rightarrow y&(x)=C_1e^{x}+C_2e^{-4x} \text{, where $C_1,C_2$ are arbitrary constants, is the general solution of the equation.}
\end{flalign*}
\textbf{Example: }Find the general solution of the differential equation
\begin{center}
    $y''+9y=0$\\
    \textbf{Solution}
\end{center}
The characteristic equation is:
\begin{flalign*}
    \ \ &t^2+9=0&&\\
    \iff&t=\pm 3i\\
    \Rightarrow y&(x)=C_1e^{0x}\cos(3x)+C_2e^{0x}\sin(3x)=C_1\cos(3x)+C_2\sin(3x) \text{ ($C_1,C_2$ are arbitrary constants)}
\end{flalign*}
Finally, $y=C_1\cos(3x)+C_2\sin(3x)$, where $C_1,C_2$ are arbitrary constants, is the general solution of the equation.\\
\textbf{Example: }Find the general solution of the differential equation
\begin{center}
    $y''-2y'+3y=0$\\
    \textbf{Solution}
\end{center}
The characteristic equation is:
\begin{flalign*}
    \ \ &t^2-2t+3=0&&\\
    \iff& t=1\pm i\sqrt{2}\\
    \Rightarrow y&(x)=C_1e^{x}\cos(x\sqrt{2})+C_2e^{x}\sin(x\sqrt{2}) \text{, where $C_1,C_2$ are constants, is the general solution of the equation.}
\end{flalign*}
\textbf{Example: }Find the general solution of the differential equation
\begin{center}
    $y''+4y'+4y=0$\\
    \textbf{Solution}
\end{center}
The characteristic equation is:
\begin{flalign*}
    \ \ &t^2+4t+4=0&&\\
    \iff&t=-2&&\\
    \Rightarrow y&(x)=C_1e^{-2x}+C_2xe^{-2x} \text{, where $C_1,C_2$ are arbitrary constants, is the general solution of the equation.}
\end{flalign*}
\newpage
\section{Non-Homogeneous Second-Order Linear Differential Equations}
\subsection{Definition and Theorem}
\begin{mybox} 
    \textbf{[Definition]}\\
    The standard form of a \textbf{Non-Homogeneous Second-Order Linear Differential Equation} with constant-coefficients:
    \begin{equation}
        y''+ay'+by=F(x)
    \end{equation}
    where $a,b$ are constants.
    The equation
    \begin{equation}
        y''+ay'+by=0
    \end{equation}
    is called the \textbf{associated homogeneous equation} to the non-homogeneous equation (7).
\end{mybox}
\begin{mybox}
    \textbf{[Theorem]}\\
    If $u_p$ is a particular solution of
    \begin{equation*}
        y''+ay'+by=f(x)
    \end{equation*}
    and $u_q$ is a particular solution of
    \begin{equation*}
        y''+ay'+by=g(x)
    \end{equation*}
    then $u_p+u_q$ is a solution of
    \begin{equation*}
        y''+ay'+by=f(x)+g(x)
    \end{equation*}
\end{mybox}
\begin{mybox}
    \textbf{[Theorem]}\\
    The \textbf{general solution} to the non-homogeneous differential equation (7) has the form:
    \begin{equation*}
        y(x)=y_c(x)+y_p(x)
    \end{equation*}
    where the \textbf{complementary solution} $y_c(x)$ is the general solution of the associated homogeneous equation (8) and $y_p(x)$ is any \textbf{particular solution} of the non-homogeneous equation (7).
\end{mybox}
\subsection{The Method of Undetermined Coefficients}
In this section, we will study the technique for obtaining a particular solution $y_p$ to the equation: \\(7)\ \ \ \ \ \ \ \ \ \ \ \ \ \ \ \ \ \ \ \ \ \ \ \ \ \ \ $y''+ay'+by=F(x)$\\
when F(x) is one of the following forms: a power of x, a natural exponential function, a sine or cosine, sums or products of those functions above.\\
The technique is choosing a form of $y_p$ similar to $F(x)$, but with unknown coefficients. These coefficients will be determined by substituting that $y_p$ into the given equation.
\begin{mybox}
    \textbf{[The choice rules]}\\
    \textbf{(a) Basic rule:}
    \begin{itemize}
        \item If $F(x)=Cx^k (k=0,1,2,...)$, then choose $y_p(x)=A_0+A_1x+A_2x^2+...+A_kx^k.$
        \item If $F(x)=Ce^{\alpha x}$, then choose $y_p(x)=Ae^{\alpha x}$
        \item If $F(x)=C\sin\omega x$ or $F(x)=C\cos\omega x$ or $F(x)=C_1\cos\omega x+C_2\sin\omega x$, \\then choose $y_p(x)=A_0\sin\omega x+A_1\cos\omega x$
    \end{itemize}
    \textbf{(b) Sum/product rule:} If F(x) is sums or products of equations above, then $y_p(x)$ is sums or products of corresponding functions in
those equations (Eg: $F(x)=xe^x,5e^x-\sin2x,5x^4e^{2x}\cos \displaystyle\frac{\pi}{2}x$).\\
    \textbf{(c) Modification rule:} If a term in our choice for $y_p(x)$ is a solution of the associated homogeneous equation, then multiply this term by $x$ (or $x^2$ if this solution corresponds to a double root of the characteristic equation of the associated homogeneous equation).
\end{mybox}
\begin{center}
\begin{tabular}{lll}
  \toprule
  $F(x)$     & Condition & $y_p(x)$   \\
  \midrule
  $Ce^\alpha x$ & $\alpha$ is not a root of the characteristic equation  & $Ae^{\alpha x}$  \\
            & $\alpha$ is a single root of the characteristic equation   & $Axe^{\alpha x}$ \\
            & $\alpha$ is a double root of the characteristic equation & $Ax^2e^{\alpha x}$  \\
  \addlinespace  
  $C_1\cos\omega x+C_2\sin\omega x$  &$\alpha\pm\omega i$ are not complex roots of the
characteristic equation   & $A_0\cos\omega x+A_1\sin\omega x$  \\
            &$\alpha\pm\omega i$ are complex roots of the
characteristic equation   & $x(A_0\cos\omega x+A_1\sin\omega x)$  \\
  \addlinespace
  $px^2+qx+m$  & 0 is not a root of the
characteristic equation   & $A_0x^2+A_1x+A_2$  \\
            & 0 is a single root of the
characteristic equation   & $x(A_0x^2+A_1x+A_2)$  \\
            & 0 is a double root of the
characteristic equation   & $x^2(A_0x^2+A_1x+A_2)$ \\
  \bottomrule
\end{tabular}
\end{center}
\begin{center}
\textit{Some example of choosing $y_p(x)$}
\end{center}
\textbf{Example: }Find the general solution of the equation
\begin{flalign*}
    &&y''- 2y' - 3y = 1 - x^2&&(*)
\end{flalign*}
\begin{center}
    \textbf{Solution}
\end{center}
First, we have to find $y_c$ by solving the associated homogeneous equation $y''-2y'-3y=0$\\
The characteristic equation is:
\begin{flalign*}
    &t^2-2t-3=0&&\\
    \iff&\left[\begin{matrix}
        t=&3\\
        t=&-1
    \end{matrix}
    \right.\\
    \Rightarrow y_c&=C_1e^{3x}+C_2e^{-x} \text{ ($C_1,C_2$ are arbitrary constants)}
\end{flalign*}
Then we choose $y_p(x)$ by looking at the right-hand side of the given equation. Because $-x^2+1$ is a quadratic polynomial and 0 is not a root of the characteristic equation, we choose $y_p=A_0x^2+A_1x+A_2$.\\
$\Rightarrow y_p'=2A_0x+A_1\\
\Rightarrow y_p''=2A_0$\\
After that, we substitute $y_p,y_p'$ and $y_p''$ into the given equation (*)
\begin{flalign*}
(*)\iff&&2A_0-2(2A_0x+A_1)-3(A_0x^2+A_1x+A_2)&=1-x^2&&\\
    \iff&&-3A_0x^2+(-4A_0-3A_1)x+(2A_0-2A_1-3A_2)&=-x^2+1&&
\end{flalign*}
\begin{flalign*}
    \Rightarrow\left\{
    \begin{matrix}
        -3A_0&=&-1\\
        -4A_0-3A_1&=&0\\
        2A_0-2A_1-3A_2&=&1
    \end{matrix}
    \right.
    \iff\left\{
    \begin{matrix}
        A_0&=&\displaystyle\frac{1}{3}\\\\
        A_1&=&\displaystyle\frac{-4}{9}\\\\
        A_2&=&\displaystyle\frac{5}{27}
    \end{matrix}
    \right.&&
\end{flalign*}
$\Rightarrow y_p=\displaystyle\frac{1}{3}x^2-\frac{4}{9}x+\frac{5}{27}\\
\Rightarrow y=y_c+y_p=C_1e^{3x}+C_2e^{-x}+\frac{1}{3}x^2-\frac{4}{9}x+\frac{5}{27}$, where $C_1,C_2$ are arbitrary constants, is the general solution of the equation.\\
\textbf{Example: } Find the general solution of the equation
\begin{flalign*}
    &&y'' + 4y' + 4y = e^{-x}\cos x&&(*)\\
    &&\textbf{Solution}&&\ 
\end{flalign*}
The characteristic equation is:
\begin{flalign*}
    &t^2+4t+4=0&&\\
    \iff& t=-2&&\\
    \Rightarrow y_c&=C_1e^{-2x}+C_2xe^{-2x} \text{ ($C_1,C_2$ are arbitrary constants)}
\end{flalign*}
Because $e^{-x}$ and $\cos x$ are not solutions of the associated homogeneous equation, we do not need to multiply $x$ to $y_p$.\\
Choose $y_p=e^{-x}(A_0\cos x+A_1\sin x)=A_0e^{-x}\cos x+A_1e^{-x}\sin x$
\begin{flalign*}
\Rightarrow y_p'=&-A_0e^{-x}\cos x-A_0e^{-x}\sin x-A_1e^{-x}\sin x+A_1e^{-x}\cos x\\
\Rightarrow y_p''=&A_0e^{-x}\cos x+A_0e^{-x}\sin x+A_0e^{-x}\sin x-A_0e^{-x}\cos x+A_1e^{-x}\sin x-A_1e^{-x}\cos x\\-A_1e^{-x}&\cos x-A_1e^{-x}\sin x\\
    =&2A_0e^{-x}\sin x-2A_1e^{-x}\cos x    \\
    (*) \iff&2A_0e^{-x}\sin x-2A_1e^{-x}\cos x +4(A_0e^{-x}\cos x-A_0e^{-x}\sin x-A_1e^{-x}\sin x+A_1e^{-x}\cos x)\\
    &+4(A_0e^{-x}\cos x+A_1e^{-x}\sin x)=e^{-x}\cos x\\
    \iff&-2A_0e^{-x}\sin x+2A_1e^{-x}\cos x+8A_0e^{-x}\cos x=e^{-x}\cos x\\
    \iff&(2A_1+8A_0)e^{-x}\cos x-2A_0e^{-x}\sin x=e^{-x}\cos x
\end{flalign*}
\begin{flalign*}
    \Rightarrow\left\{
    \begin{matrix}
        2A_1+8A_0=&1\\
        -2A_0=&0
    \end{matrix}
    \right.
    \iff\left\{
    \begin{matrix}
        A_0&=&0\\
        A_1&=&\displaystyle\frac{1}{2}\\
    \end{matrix}
    \right.&&
\end{flalign*}
$\Rightarrow y_p=\displaystyle\frac{1}{2}e^{-x}\sin x\\
\Rightarrow y=y_c+y_p=C_1e^{-2x}+C_2xe^{-2x}+\displaystyle\frac{1}{2}e^{-x}\sin x$, where $C_1,C_2$ are arbitrary constants, is the general solution of the equation.\\
\textbf{Example:} Solve the IVP
\begin{flalign*}
    &&y''-6y' + 9y = e^{3x}\ (*),\ y(0)=2,\ y'(0)=8&&
\end{flalign*}
\begin{center}
    \textbf{Solution}
\end{center}
The characteristic equation is:
\begin{flalign*}
    &t^2-6t+9=0&&\\
    \iff& t=3&&\\
    \Rightarrow y_c&=C_1e^{3x}+C_2xe^{3x} \text{ ($C_1,C_2$ are arbitrary constants)}
\end{flalign*}
Because $t=3$ is a double root of the characteristic equation, then $e^{3x}$ is a solution of the associated homogeneous equation (when $C_1=1,C_2=0 \Rightarrow y_c=e^{3x})$, then choose $y_p=Ax^2e^{3x}.$\\
$\Rightarrow y'_p=2Axe^{3x}+3Ax^2e^{3x}=Ae^{3x}(2x+3x^2)\\
\Rightarrow y''_p=2Ae^{3x}+6Axe^{3x}+6Axe^{3x}+9Ax^2e^{3x}=Ae^{3x}(2+6x+6x+9x^2)\\
(*)\iff 2Ae^{3x}=e^{3x}\\
\Rightarrow 2A=1\\
\iff A=\displaystyle\frac{1}{2}\\
\Rightarrow y_p=\frac{1}{2}x^2e^{3x}\\
\Rightarrow y=y_c+y_p=C_1e^{3x}+C_2xe^{3x}+\frac{1}{2}x^2e^{3x}$ ($C_1,C_2$ are arbitrary constants)\\
$\Rightarrow y'=3C_1e^{3x}+C_2e^{3x}+3C_2xe^{3x}+xe^{3x}+\displaystyle\frac{3}{2}x^2e^{3x}$\\
We have: \\
$y(0)=2 \Rightarrow C_1=2$\\
$y'(0)=8 \Rightarrow 3C_1+C_2=8 \iff C_2=2$\\
Finally, $y=2e^{3x}+2xe^{3x}+\displaystyle\frac{1}{2}x^2e^{3x}$ is the solution of the IVP.\\\\\\\\\\
\textbf{Example: }Find the general solution of the equation
\begin{flalign*}
    &&y''+16y=\cos 2x+\sin 2x&&(*)\\
    &&\textbf{Solution}&&
\end{flalign*}
The characteristic equation is:
\begin{flalign*}
    &t^2+16=0&&\\
    \iff&t=\pm4i&&\\
    \Rightarrow&y_c=C_1e^{0x}\cos 4x+C_2e^{0x}\sin 4x=C_1\cos 4x+C_2\sin 4x \text{ ($C_1,C_2$ are arbitrary constants)}
\end{flalign*}
Because $\cos x$ and $\sin x$ are not a solution of the associated homogeneous equation, the roots of the characteristic equation do not have the form $t=\alpha\pm 2i$ (they are $t=\pm 4i$), then choose $y_p=A_0\cos 2x+A_1\sin 2x.\\
\Rightarrow y'_p=-2A_0\sin 2x+2A_1\cos 2x\\
\Rightarrow y''_p=-4A_0\cos 2x-4A_1\sin 2x$
\begin{flalign*}
    (*)\iff&&-4A_0\cos 2x-4A_1\sin 2x+16A_0\cos 2x+16A_1\sin 2x&=\cos 2x+\sin 2x&&\\
    \iff&&12A_0\cos 2x+12A_1\sin 2x&=\cos 2x+\sin 2x&&
\end{flalign*}
$\Rightarrow$
    $\left\{\begin{matrix}
        12A_0=&1\\
        12A_1=&1
    \end{matrix}
    \right.
    \iff \left\{\begin{matrix}
        A_0=&\displaystyle\frac{1}{12}\\\\
        A_1=&\displaystyle\frac{1}{12}
    \end{matrix}
    \right.\\
    \Rightarrow y_p=\displaystyle\frac{1}{12}\cos 2x+\frac{1}{12}\sin 2x\\
    \Rightarrow y=y_c+y_p=C_1\cos 4x+C_2\sin 4x+\displaystyle\frac{1}{12}\cos 2x+\frac{1}{12}\sin 2x
    $, where $C_1,C_2$ are arbitrary constants, is the general solution of the equation.\\
\textbf{Example: } Find the general solution of the equation
\begin{flalign*}
    &&y''+16y=\cos 4x&&(*)\\
    &&\textbf{Solution}&&
\end{flalign*}
The characteristic equation is:
\begin{flalign*}
    &t^2+16=0&&\\
    \iff&t=\pm4i&&\\
    \Rightarrow&y_c=C_1\cos 4x+C_2\sin 4x \text{ ($C_1,C_2$ are arbitrary constants)}
\end{flalign*}
Because $\cos 4x$ is a solution of the associated homogeneous equation (when $C_1=1,C_2=0\Rightarrow y_c=\cos 4x$), the roots of the characteristic equation have the form $t=\alpha\pm 4i$, then choose:
\begin{flalign*}
    y_p&=x(A_0\cos 4x+A_1\sin 4x)=A_0x\cos 4x+A_1x\sin 4x&\\
    \Rightarrow y'_p&=A_0\cos 4x-4A_0x\sin 4x+A_1\sin 4x+4A_1x\cos 4x\\
    \Rightarrow y''_p&=-4A_0\sin 4x-4A_0\sin 4x-16A_0x\cos 4x+4A_1\cos 4x+4A_1\cos 4x-16A_1x\sin 4x&&\\
    &=-8A_0\sin 4x+8A_1\cos 4x-16A_0x\cos 4x-16A_1x\sin 4x&&\\
    (*)\iff&-8A_0\sin 4x+8A_1\cos 4x-16A_0x\cos 4x-16A_1x\sin 4x+16A_0x\cos 4x+16A_1x\sin 4x=\cos 4x&&\\
    \iff&-8A_0\sin 4x+8A_1\cos 4x=\cos 4x&&\\
    \iff&\left\{
        \begin{matrix}
            -8A_0=0\\
            8A_1=1
        \end{matrix}
    \right.\\
    \iff&\left\{
        \begin{matrix}
            A_0=0\\
            A_1=\displaystyle\frac{1}{8}
        \end{matrix}
    \right.
\end{flalign*}
$\Rightarrow y_p=\displaystyle\frac{1}{8}x\sin 4x\\
\Rightarrow y=y_c+y_p=C_1\cos 4x+C_2\sin 4x+\displaystyle\frac{1}{8}x\sin 4x$, where $C_1,C_2$ are arbitrary constants, is the general solution of the equation.
\subsection{The Method of Variation-of-Parameters}
In this section, we will study the technique for obtaining a particular solution $y_p$ to the equation
\begin{equation*}
    y'' + ay' + by = F (x)
\end{equation*}
when you can not use the method of undetermined coefficients (section 3.2) to find $y_p(x)$.
\begin{mybox}
    Consider a differential equation 
    \begin{equation}
        y'' + ay'+ by = F(x)
    \end{equation}
where $a,b,F(x)$ are (at least) continuous on an interval $I$.\\
Let $y_1(x), y_2(x)$ be linearly independent solutions to the associated
homogeneous equation
    \begin{equation*}
        y''+ay'+b=0
    \end{equation*} on $I$.\\
    Then a particular solution to Equation (9) is 
    \begin{equation*}
        y_p(x) = C_1(x)y_1(x) + C_2(x)y_2(x),
    \end{equation*}
    where  $C_1(x),C_2(x)$ satisfy $\left\{
        \begin{matrix}
            C_1'y_1+C_2'y_2&=&0\\
            C_1'y_1'+C_2'y_2'&=&F(x)
        \end{matrix}
    \right.$
\end{mybox}
To solve the system of equations above, you should use the Cramer's Rule, if you do not remember what Cramer's Rule is, then look back: Slide 64 $\rightarrow$ 68 - Lecture 3 - MAT1.002 Linear Algebra.\\(Link: \url{https://drive.google.com/file/d/1_JsNbZv1O7VIzdVLVdBrA-ZLrMBG0xvi/view?usp=sharing})\\
\textbf{Example:} Find the general solution to the equation
\begin{flalign*}
    &&y''+y=\tan x&&(*)\\
    &&\textbf{Solution}&&
\end{flalign*}
The characteristic equation is
\begin{flalign*}
    &t^2+1=0 \iff t=\pm i&&\\
    \Rightarrow&y_1(x)=\cos x,\ y_2(x)=\sin x\\
    \Rightarrow&y_c(x)=C_1\cos x+C_2\sin x \text{ ($C_1,C_2$ are arbitrary constants)}
\end{flalign*}
Let $y_p(x)=C_1(x)y_1(x)+C_2(x)y_2(x)=C_1(x)\cos x+C_2(x)\sin x$,\\
where $C_1(x),C_2(x)$ satisfy $\left\{
        \begin{matrix}
            C_1'y_1+C_2'y_2&=&0\\
            C_1'y_1'+C_2'y_2'&=&F(x)
        \end{matrix}
    \right.
    \iff\left\{
        \begin{matrix}
            C_1'\cos x+C_2'\sin x&=&0\\
            -C_1'\sin x+C_2'\cos x&=&\tan x
        \end{matrix}
    \right.$\\
Apply Cramer's Rule, we have:
\begin{flalign*}
    &C_1'=\displaystyle\frac{\begin{vmatrix}
        0&\sin x\\
        \tan x&\cos x
    \end{vmatrix}}{\begin{vmatrix}
        \cos x&\sin x\\
        -\sin x&\cos x
    \end{vmatrix}}
    =\frac{-\sin x\tan x}{\cos^2x+\sin^2 x}=-\sin x\tan x=\frac{-\sin^2 x}{\cos x}&&\\
    &C_2'=\displaystyle\frac{\begin{vmatrix}
        \cos x&0\\
        -\sin x&\tan x
    \end{vmatrix}}{\begin{vmatrix}
        \cos x&\sin x\\
        -\sin x&\cos x
    \end{vmatrix}}=\cos x\tan x=\sin x&&
\end{flalign*}
\begin{flalign*}
    \Rightarrow C_1=&\displaystyle\int\frac{-\sin^2 x}{\cos x}dx=\int\frac{-(1-\cos^2x)}{\cos x}dx=\int\left(\cos x-\frac{1}{\cos x}\right)dx=\int(\cos x-\sec x)dx&&\\
    =&\sin x-\ln|\sec x+\tan x|&&\\
    C_2=&\displaystyle\int\sin x\ dx=-\cos x&&\\
    \Rightarrow y_p(x)=&(\sin x-\ln|\sec x+\tan x|)\cos x-\cos x\sin x=-\cos x.\ln|\sec x+\tan x|&&\\
    \Rightarrow y(x)=&C_1\cos x+C_2\sin x-\cos x.\ln|\sec x+\tan x|,\text{  where $C_1,C_2$ are constants, is the general solution.}
\end{flalign*}
\textbf{Example:} Find the general solution to the equation
\begin{flalign*}
    &&y'' + y = \sec x + 4e^{x}, -\displaystyle\frac{\pi}{2}<x<\frac{\pi}{2}&&
\end{flalign*}
\begin{center}
    \textbf{Solution}
\end{center}
The characteristic equation is:
\begin{flalign*}
    &t^2+1=0 \iff t=\pm i&\\
    \Rightarrow&\ y_1(x)=\cos x,\ y_2(x)=\sin x\\
    \Rightarrow&\ y_c=C_1\cos x+C_2\sin x \text{ ($C_1,C_2$ are arbitrary constants)}
\end{flalign*}
\textbf{First solution:} Let $y_p=C_1(x)\cos x+C_2(x)\sin x$, where $C_1(x),C_2(x)$ satisfy:
\begin{equation*}
    \left\{
    \begin{matrix}
        C_1'\cos x+C_2'\sin x&=&0\\
        -C_1'\sin x+C_2'\cos x&=&\sec x+4e^x
    \end{matrix}
\right.
\end{equation*}
Apply Cramer's Rule, we have:
\begin{flalign*}
    &C_1'=\displaystyle\frac{\begin{vmatrix}
        0&\sin x\\
        \sec x+4e^x&\cos x
    \end{vmatrix}}{\begin{vmatrix}
        \cos x&\sin x\\
        -\sin x&\cos x
    \end{vmatrix}}
    =\frac{-\sin x\sec x-4e^x\sin x}{\cos^2x+\sin^2 x}=-\tan x-4e^x\sin x&&\\
    &C_2'=\displaystyle\frac{\begin{vmatrix}
        \cos x&0\\
        -\sin x&\sec x+4e^x
    \end{vmatrix}}{\begin{vmatrix}
        \cos x&\sin x\\
        -\sin x&\cos x
    \end{vmatrix}}=\cos x\sec x+4e^x\cos x=1+4e^x\cos x&&
\end{flalign*}
You can integrate $C_1'$ and $C_2'$ by yourselves (using integration by parts):
\begin{flalign*}
    C_1(x)&=\ln (\cos x) - 2(\sin x -\cos x)e^x\\
    C_2(x)&=x + 2(\sin x + \cos x)e^x
\end{flalign*}
$\Rightarrow y(x) = C_1 \cos x + C_2 \sin x + 2e^x + \cos x \ln(\cos x) + x\sin x$, where $C_1,C_2$ are arbitrary constants, is the general solution to the equation.\\
\textbf{Second solution:} Using the first theorem in section 3.1 - Page 7.\\
We can consider 2 equations: 
\begin{flalign*}
    y'' + y = \sec x\ (1)\text{ and } y'' + y = 4e^{x}\ (2)
\end{flalign*}
The sum of their particular solutions $y_{p1}$ and $y_{p2}$ will be a particular solution to the initial equation.\\
For equation (1), use the method of variation-of-parameters.\\
For equation (2), it can be solved by the method of undetermined coefficient. \\
Consider $y'' + y = \sec x$\\
Let $y_{p1}=C_1(x)\cos x+C_2(x)\sin x$, where $C_1(x),C_2(x)$ satisfy:
\begin{equation*}
    \left\{
    \begin{matrix}
        C_1'\cos x+C_2'\sin x&=&0\\
        -C_1'\sin x+C_2'\cos x&=&\sec x
    \end{matrix}
\right.
\end{equation*}
Apply Cramer's Rule, we have:
\begin{flalign*}
    &C_1'=\displaystyle\frac{\begin{vmatrix}
        0&\sin x\\
        \sec x&\cos x
    \end{vmatrix}}{\begin{vmatrix}
        \cos x&\sin x\\
        -\sin x&\cos x
    \end{vmatrix}}
    =\frac{-\sin x\sec x}{\cos^2x+\sin^2 x}=-\tan x\ \Rightarrow C_1(x)=\int(-\tan x)dx=\ln|\cos x|=\ln(\cos x)\ (\displaystyle |x|<\frac{\pi}{2})&\\
    &C_2'=\displaystyle\frac{\begin{vmatrix}
        \cos x&0\\
        -\sin x&\sec x
    \end{vmatrix}}{\begin{vmatrix}
        \cos x&\sin x\\
        -\sin x&\cos x
    \end{vmatrix}}=\cos x\sec x=1 \ \Rightarrow C_2(x)=\int 1dx=x&&
\end{flalign*}
$\Rightarrow y_{p1}=\cos x\ln(\cos x)+x\sin x$\\
Consider $y'' + y = 4e^x$ (2)\\
Choose $y_{p2}=Ce^x\\
\Rightarrow y_p'=y''_p=Ce^x\\
(2) \iff Ce^x+Ce^x=4e^x\\
\iff C=2\\
\Rightarrow y_{p2}=2e^x\\
\Rightarrow y_p=y_{p1}+y_{p2}=\cos x\ln(\cos x)+x\sin x+2e^x$ is a particular solution to the initial equation.
\section{Series Solutions to Linear Differential Equations}
\textbf{Caution:} This is a hard section, learn it if you want to get very high scores!!!
\subsection{Definition and Theorem}
The number $R$ is called the radius of convergence and $(-R, R)$ is
called the interval of convergence of the power series.
\begin{mybox}
    \textbf{[Theorem: Radius of convergence]}\\
    Suppose for $\displaystyle\sum_{n=0}^\infty a_nx^n$ we have
    \begin{flalign*}
        \displaystyle\lim_{n \rightarrow \infty} \left|\frac{a_{n+1}}{a_n}\right|=\alpha.
    \end{flalign*}
    \begin{itemize}
            \item If $\alpha = 0$, then $R = \infty$, the series converges for all $x$.
            \item If $\alpha =\infty$, then $R = 0$, the series converges at $x=0$.
            \item If $\alpha>0$, then $R=\displaystyle\frac{1}{\alpha}=\lim_{n\rightarrow\infty}\left|\frac{a_n}{a_{n+1}}\right|$
    \end{itemize}
\end{mybox}
\begin{mybox}
    \textbf{[Definition]}\\
    A function is said to be \textbf{analytic} at $x = a$, if it can be represented in the form of a convergent power series $\displaystyle\sum_{n=0}^\infty a_n(x-a)^n$ with nonzero radius
of convergence.\\
If functions $f(x)$ and $g(x)$ are analytic at $x = a$, then $f(x) \pm g(x), f(x)g(x)$ and $\displaystyle\frac{f(x)}{g(x)}$ (provided that $g(a)\ne 0$) are analytic at $x=a$, too.
\end{mybox}
\begin{mybox}
    \textbf{[Definition]}
    \begin{itemize}
        \item The point $x = a$ is called an \textbf{ordinary point} of the differential equation
            \begin{flalign*}
                y'' + p(x)y' + q(x)y = 0,
            \end{flalign*}
        if $p$ and $q$ are both analytic at $x = a$.
        \item Any point which is not an ordinary point is called \textbf{singular point} of the equation.
    \end{itemize}
\end{mybox}
\textbf{Note: }You should pay attention to everything after this page.
\newpage
\begin{mybox}
    \textbf{[Differentiation of power series] \color{red}(IMPORTANT)}\\
    Suppose that $\displaystyle\sum_{n=0}^\infty a_nx^n$ has radius of convergence R, and let
    \begin{flalign*}
        f(x)=\displaystyle\sum_{n=0}^\infty a_nx^n,\ |x|<R
    \end{flalign*}
    Then $f(x)$ can be differentiated an arbitrary number of times on the
interval $|x| < R$. The derivatives can be obtained by termwise
differentiation:
    \begin{flalign*}
        f'(x)=&\displaystyle\sum_{n=0}^\infty na_nx^{n-1}=\sum_{n=1}^\infty na_nx^{n-1}\\
        f''(x)=&\sum_{n=0}^\infty n(n-1)a_nx^{n-2}=\sum_{n=2}^\infty n(n-1)a_nx^{n-2}
    \end{flalign*}
    and so on for higher-order derivatives.
\end{mybox}
\subsection{Solution technique}
\begin{mybox}
    Let $p(x)$ and $q(x)$ be analytic at $x = 0$, and suppose that their power series expansions are valid for |x| < R.\\
    Then the general solution to the differential equation
    \begin{equation}
         y''+p(x)y'+q(x)y=0
    \end{equation}
    can be represented in the form of a power series
    \begin{flalign*}
        y(x)=\displaystyle\sum_{n=0}^\infty a_nx^n
    \end{flalign*}
    with the radius of convergence at least $R$.\\
    The coefficients in this series solution can be determined in terms
of $a_0$ and $a_1$ by directly substituting $y(x)$ into the equation (10).\\
    The resulting solution is of the form
    \begin{flalign*}
        y(x)=a_0y_1(x)+a_1y_2(x),
    \end{flalign*}
    where $y_1$ and $y_2$ are linearly independent solutions to the given equation on the interval of existence.
\end{mybox}
If the initial conditions $y(a) = A$, $y'(a) = B$ are imposed, then $a_0 = A, a_1 = B$. \\
\textbf{Example: }Find the series solution of the equation
\begin{flalign*}
     &&y'' + y' + x^2y = 0&&(*)
\end{flalign*}
\begin{center}
    \textbf{Solution}
\end{center}
Assume the series solution to the equation is 
\begin{flalign*}
    &y(x)=\displaystyle\sum_{n=0}^\infty a_nx^n\\
    \Rightarrow&\ y'(x)=\displaystyle\sum_{n=0}^\infty na_nx^{n-1}=\sum_{n=1}^\infty na_nx^{n-1}\\
    \Rightarrow&\ y''(x)=\sum_{n=0}^\infty n(n-1)a_nx^{n-2}=\sum_{n=2}^\infty n(n-1)a_nx^{n-2}
\end{flalign*}
\begin{flalign*}
    (*)\iff&&\ \displaystyle\sum_{n=2}^\infty n(n-1)a_nx^{n-2}+\sum_{n=1}^\infty na_nx^{n-1}+x^2\sum_{n=0}^\infty a_nx^n&=0&&\\
    \iff&&\ \displaystyle\sum_{n=2}^\infty n(n-1)a_nx^{n-2}+\sum_{n=1}^\infty na_nx^{n-1}+\sum_{n=0}^\infty a_nx^{n+2}&=0&&\\
    \iff&&\ \displaystyle\sum_{n=0}^\infty (n+2)(n+1)a_{n+2}x^{n}+\sum_{n=0}^\infty (n+1)a_{n+1}x^{n}+\sum_{n=2}^\infty a_{n-2}x^n&=0&&\\
    \iff&&\ (2a_2+6a_3x+12a_4x^2+20a_5x^3+...)+(a_1+2a_2x+3a_3x^2+4a_4x^3+...)+(a_0x^2+a_1x^3+...)&=0&&\\
    \iff&&\ (2a_2+a_1)+(6a_3+2a_2)x+(12a_4+3a_3+a_0)x^2+(20a_5+4a_4+a_1)x^3+...&=0&&
\end{flalign*}
    Therefore, we have:
\begin{flalign*}
    &&2a_2+a_1=0 &\iff a_2=-\displaystyle\frac{1}{2}a_1&&\\
    &&6a_3+2a_2=0 &\iff a_3=\frac{1}{6}a_1&&\\
    &&12a_4+3a_3+a_0=0 &\iff a_4=-\frac{1}{24}a_1-\frac{1}{12}a_0&&\\
    &&20a_5+4a_4+a_1=0 &\iff a_5=-\frac{1}{24}a_1+\frac{1}{60}a_0&&\\
    &&&\ \ \ ...&&
\end{flalign*}
\begin{flalign*}
    \Rightarrow y(x)&=\displaystyle\sum_{n=0}^\infty a_nx^n&&\\
    &=a_0+a_1x+a_2x^2+a_3x^3+a_4x^4+a_5x^5+...&&\\
    &=a_0+a_1x-\frac{1}{2}a_1x^2+\frac{1}{6}a_1x^3-\frac{1}{24}a_1x^4-\frac{1}{12}a_0x^4-\frac{1}{24}a_1x^5+\frac{1}{60}a_0x^5+...&&\\
    &=a_0(1-\frac{1}{12}x^4+\frac{1}{60}x^5+...)+a_1(x-\frac{1}{2}x^2-\frac{1}{24}x^4-\frac{1}{24}x^5+...)&&
\end{flalign*}
Finally, the general solution of the equation is $y=a_0(1-\displaystyle\frac{1}{12}x^4+\frac{1}{60}x^5+...)+a_1(x-\frac{1}{2}x^2-\frac{1}{24}x^4-\frac{1}{24}x^5+...)$, which consists of two solutions: $y_1(x)=1-\displaystyle\frac{1}{12}x^4+\frac{1}{60}x^5+...$ and $y_2(x)=x-\displaystyle\frac{1}{2}x^2-\frac{1}{24}x^4-\frac{1}{24}x^5+...$.\\
\textbf{Example: }Find the series solution to the equation
\begin{flalign*}
    &&y''-2xy'+y=0&&(*)
\end{flalign*}
\begin{center}
    \textbf{Solution}
\end{center}
Assume the series solution to the equation is
\begin{flalign*}
    &y(x)=\displaystyle\sum_{n=0}^\infty a_nx^n\\
    \Rightarrow&\ y'(x)=\displaystyle\sum_{n=0}^\infty na_nx^{n-1}=\sum_{n=1}^\infty na_nx^{n-1}\\
    \Rightarrow&\ y''(x)=\sum_{n=0}^\infty n(n-1)a_nx^{n-2}=\sum_{n=2}^\infty n(n-1)a_nx^{n-2}
\end{flalign*}
\begin{flalign*}
    (*)\iff&&\ \displaystyle\sum_{n=2}^\infty n(n-1)a_nx^{n-2}-2x\sum_{n=1}^\infty na_nx^{n-1}+\sum_{n=0}^\infty a_nx^n&=0&&\\
    \iff&&\ \displaystyle\sum_{n=0}^\infty (n+2)(n+1)a_{n+2}x^{n}-\sum_{n=1}^\infty 2na_nx^{n}+\sum_{n=0}^\infty a_nx^n&=0&&\\
    \iff&&\ (2a_2+6a_3x+12a_4x^2+20a_5x^3+...)-(2a_1x+4a_2x^2+6a_3x^3+...)+(a_0+a_1x+a_2x^2+a_3x^3+...)&=0&&\\
    \iff&&\ (2a_2+a_0)+(6a_3-2a_1+a_1)x+(12a_4-4a_2+a_2)x^2+(20a_5-6a_3+a_3)x^3+...&=0&&
\end{flalign*}
Therefore, we have:
\begin{flalign*}
    &&2a_2+a_0=0 &\iff a_2=-\displaystyle\frac{1}{2}a_0&&\\
    &&6a_3-2a_1+a_1=0 &\iff a_3=\frac{1}{6}a_1&&\\
    &&12a_4-4a_2+a_2=0 &\iff a_4=-\frac{1}{8}a_0&&\\
    &&20a_5-6a_3+a_3=0 &\iff a_5=\frac{1}{24}a_1&&\\
    &&&\ \ \ ...&&
\end{flalign*}
\begin{flalign*}
    \Rightarrow y(x)&=\displaystyle\sum_{n=0}^\infty a_nx^n&&\\
    &=a_0+a_1x+a_2x^2+a_3x^3+a_4x^4+a_5x^5+...&&\\
    &=a_0+a_1x-\frac{1}{2}a_0x^2+\frac{1}{6}a_1x^3-\frac{1}{8}a_0x^4+\frac{1}{24}a_1x^5+...&&\\
    &=a_0(1-\frac{1}{2}x^2-\frac{1}{8}x^4+...)+a_1(x+\frac{1}{6}x^3+\frac{1}{24}x^5+...)&&
\end{flalign*}
Finally, the general solution of the equation is $y=a_0(1-\displaystyle\frac{1}{2}x^2-\frac{1}{8}x^4+...)+a_1(x+\frac{1}{6}x^3+\frac{1}{24}x^5+...)$, which consists of two solutions: $\displaystyle y_1(x)=1-\frac{1}{2}x^2-\frac{1}{8}x^4+...$ and $\displaystyle y_2(x)=x+\frac{1}{6}x^3+\frac{1}{24}x^5+...$.\\
\newpage
\begin{center}
    \textbf{FEEDBACK FOR THIS DOCUMENT}\\
    \includegraphics[width=0.5\linewidth]{qr.png}\\
    \textbf{Please scan this QR code...}\\
    \textbf{Or click this link:} \url{https://forms.gle/qmMhmKmVWHVKNAaG6} \\
    to give us feedback on this document!!!
\end{center}
Please spend a little time to send your feedback. Although it is just a simple work, it has great significance to us. Your feedback can help us improve our future documents and it let us know whether you understand what we have written in this document. Thank you so much for using our documents!\\
\begin{center}
    Good luck with your exam!
\end{center}
\end{document}